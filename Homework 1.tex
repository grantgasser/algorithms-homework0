\documentclass[12pt]{article}
\usepackage{geometry}
\geometry{letterpaper, margin=1in}
\usepackage{nopageno}
\pagestyle{empty}
\usepackage{color}
\usepackage{array}

% Header, put stuff that impacts the whole document here
\NeedsTeXFormat{LaTeX2e}
\title{Introduction to Algorithms\\CSI 3344\\ Homework 1}
\author{Grant Gasser} 

\begin{document}
\maketitle 

Prepare a LaTex file with each problem beginning on a fresh page.
Use the Upload Site to submit your solution. These problems are a review of material from earlier courses that will serve us well in CSI 3344:

\begin{enumerate}
\newpage
\item In each of the following situations, indicate whether $f(n)  \in  O(g(n))$, or $f(n)  \in \Omega$(g(n)), or both (in which case $f(n)  \in \Theta$(g(n))).


	\renewcommand{\arraystretch}{1.5}
	\begin{tabular}{|c|c|c|}\hline
		f(n) &  g(n) & Answer  \\ \hline
		$n - 100$ &  $n - 200$  & $f(n) \in \Theta(g(n))$ \\ \hline
		$n^\frac{1}{2} $ &  $n^\frac{2}{3} $  & $f(n) \in O(g(n))$\\ \hline
		$100n + log(n) $ &  $ n + (log (n))^2$  & $f(n) \in \Theta(g(n))$ \\ \hline
		$ n~log(n) $ &  $ 10n~log (10n)$  &$f(n) \in \Theta(g(n))$ \\ \hline
		$log (2n) $ &  $ log (3n)$  &$f(n) \in \Theta(g(n))$ \\ \hline
		$10 log(n)  $ &  $ log(n^2)$  & $f(n) \in \Theta(g(n))$\\ \hline
		$n^{1.01}  $ &  $ n log^2 n$ & $f(n) \in \Omega(g(n))$\\ \hline
		$\frac{n^2}{log (n)} $ &  $n(log(n))^2 $  & $f(n) \in \Omega(g(n))$\\ \hline
		$n^{0.1}  $ &  $ (log(n))^{10}$  & $f(n) \in \Omega(g(n))$\\ \hline
		$(log ~n)^{log ~n} $ &  $ \frac{n}{log ~n}$  & $f(n) \in \Omega(g(n))$\\ \hline
		$\sqrt{n} $ &  $  (log ~n)^3$  & $f(n) \in \Omega(g(n))$\\ \hline
		$n^{\frac{1}{2}} $ &  $ 5~log_2 ~n$  & $f(n) \in O(g(n))$\\ \hline
		$n~2^n $ &  $3^n $  & $f(n) \in O(g(n))$\\ \hline
		$2^n $ &  $ 2^{(n+1)}$  & $f(n) \in \Theta(g(n))$\\ \hline
		$n! $ &  $ 2^n $  & $f(n) \in \Omega(g(n))$\\ \hline
		$(log ~n)^{log ~n} $ &  $ 2^{(log_2 ~n)^2}$  & $f(n) \in O(g(n))$\\ \hline		
		$\sum_{i=1}^{n} i^k $ &  $ n^{k+1}$  &$f(n) \in \Theta(g(n))$ \\ \hline
	\end{tabular}


\newpage
\item Geometric series. \\
\textbf{Note: } We will use the geometric series formula: $S(n) = \frac{a-ar^{n+1}}{1-r}.$
\begin{enumerate}
	\item Give a simple upper bound on $1+2+4+ \dots +2^n$. Conclude that this sum is $O(2^n)$. 
		\begin{itemize}
			\item This can be written as $\sum_{i = 0}^{n} 2^i = 2^{n+1}-1.$ We can ignore the constant from the above table and we know that $2^{n+1} \in O(2^n).$
		\end{itemize}
	\item Do the same for $1+ \frac{1}{2 }+ \frac{1}{4} + \dots + \frac{1}{2^n} $ and conclude that the sum is O(1).
		\begin{itemize}
			\item This can be written as $\sum_{n = 0}^{\infty} (\frac{1}{2})^n = \frac{\frac{1}{2}}{1 - \frac{1}{2}}$ using the geometric series formula. In this case, $a = 1, r = \frac{1}{2}.$
			\item $\sum_{n = 0}^{\infty} (\frac{1}{2})^n = \frac{\frac{1}{2}}{1 - \frac{1}{2}} = 1$. Thus the series converges to 1. $S \in O(1)$.
		\end{itemize}
	\item By considering a more general series $S(n) = 1 + c + c^2 +  \dots + c^n$, establish the following very useful rule: in big-O terms, the sum of a geometric series is simply: \\
	\textbf{Note: } In this case, c = r, the common ratio.
\begin{enumerate}
\item the first term if the series is strictly decreasing, 
	\begin{itemize}
		\item The series is strictly decreasing if $c < 1$.
		\item Using the formula above we have $S(n) = \frac{1-c^{n+1}}{1-c}$.\\
		If $c < 1$, then $1-c < 1-c^{n+1} < 1$. Dividing by $1-c$ gives
		$1 < \frac{1-c^{n+1}}{1-c} < \frac{1}{1-c}$. \\
		$1 < S(n) < \frac{1}{1-c}$.
		\item Thus, $S(n) \in \Theta(1)$, where 1 is the first term.
	\end{itemize}
\item  the last term if the series is strictly increasing, 
	\begin{itemize}
		\item The series is strictly increasing if $c > 1$, 
		\item Using the formula above we have $S(n) = \frac{1-c^{n+1}}{1-c} = \frac{c^{n+1}-1}{c-1}$.\\
		If $c > 1$, then $c^n < c^{n+1}-1 < c^{n+1}$. Divide by 1-c gives $\frac{1}{1-c} *c^n < S(n) < \frac{c}{1-c}*c^n$
		\item Thus, $S(n) \in \Theta(c^n)$, where $c^n$ is the last term.
	\end{itemize}
\item  the number of terms if the series is unchanging.
	\begin{itemize}
		\item if $c = 1$, then $S(n) = 1 + 1 + 1 + ... + 1 = 1 + n.$ \\
		If $S(n) = n + 1$, then $S(n) \in \Theta(n).$
	\end{itemize}
\end{enumerate} 
\end{enumerate} 

\newpage
\item Proving by induction. We would like to establish the following formula for the sum of the first n odd numbers: $1+3+5 + ... +(2n-1) = n^2. $ A nice way to do this is by induction. Let S(n) be the statement above. 
\textbf{We want to prove} $\sum_{i = 1}^{n}(2n-1) = n^2.$ 

An inductive proof would have the following steps:
\begin{enumerate}
	\item  Show that S(1) is true.
     	 	\begin{enumerate}
     	\item Let $n = 1.$ Then $\sum_{i = 1}^{1}(2(1)-1) = 1^2.$\\ 
     	That gives $(2(1)-1) = 1^2.$\\ $(2-1) = 1^2.$\\ $1 = 1.$ Thus, it is true for the base case.\\	
     \end{enumerate}
     \item Assume that if S(1),...,S(k) are true, 
     \item then show S(k+1) is true.
     \begin{enumerate}
     	\item If it is true for S(1)...S(k), then $\sum_{i = 1}^{k}(2k-1) = k^2.$\\ 
     	\item Let n = k + 1. Then we have, $\sum_{i = 1}^{k+1}(2(k+1)-1) = (k+1)^2.$\\\\ $\sum_{i = 1}^{k+1}(2k+2-1) = k^2 + 2k + 1.$\\\\ $\sum_{i = 1}^{k+1}(2k+1) = k^2 + 2k + 1.$\\\\ $3 + 5 + 7 + .. + 2k-1 + 2(k+1)-1 = k^2 + 2k + 1.$\\\\ $\sum_{i = 1}^{k}(2k-1) + 2(k+1)-1 = k^2 + 2k + 1.$\\\\
     	$k^2 + 2(k+1)-1 = k^2 + 2k + 1.$ (Inductive hypothesis)\\\\ 
     	$k^2 + 2k + 1 = k^2 + 2k + 1.$ Thus, it is true for the base case and inductive hypothesis. QED.
     \end{enumerate}
\end{enumerate}

\newpage
\item Practice with $big-O ~and ~\Omega$. For some fixed positive integer c, consider the summation
$S(n)=1^c +2^c +3^c + \dots +n^c.$
\begin{enumerate}
	\item Show that S(n) is $O(n^{c+1})$. Hint: There are n terms in the series, and each is at most ...?
    \item Show that S(n) is $ \Omega(n^{c+1})$. Hint: Look just at the second half of the series.
\end{enumerate}


\newpage
\item Logarithms (base two). Recall the definition of logarithm base two: saying $p = log_2 m$ is the same as saying
$m = 2^p$. In this class, we will typically write log to mean $log_2$.
\begin{enumerate}

			\item How many bits are needed to write down a positive integer n? Give your answer in big-O notation,
			as a function of n, where n is the length of the number.
			\begin{itemize}
				\item $O(2^{n+1})$ ? 
			\end{itemize}
			
			
			\item How many times does the following piece of code print “hello”? Assume n is an integer, and that
			division rounds down to the nearest integer. Give your answer in big-O form, as a function of n.
			
			\begin{verbatim}
			    while n > 1:
			       print "hello"
			       n := n/2
			\end{verbatim}
			\begin{tabular}{|c|c|}\hline
				n &  Number of prints  \\ \hline
				0 & 0	\\ \hline
				1 & 0	\\ \hline
				2 & 1	\\ \hline
				3 & 1	\\ \hline
				4 & 2	\\ \hline
				5 & 2	\\ \hline
				6 & 2	\\ \hline
				7 & 2	\\ \hline
				8 & 3	\\ \hline
				9 & 3	\\ \hline
				8 & 3	\\ \hline
				9 & 3	\\ \hline
				10 & 3	\\ \hline
				11 & 3	\\ \hline
				12 & 3	\\ \hline
				13 & 3	\\ \hline
				14 & 3	\\ \hline
				15 & 3	\\ \hline
				16 & 4	\\ \hline
			\end{tabular}
		\vspace{5mm}
		Looks like $O(log_2(n))$

\end{enumerate}

\end{enumerate}
\end{document}






