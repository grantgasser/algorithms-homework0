\documentclass[12pt]{article}
\usepackage{geometry}
\geometry{letterpaper, margin=1in}
\usepackage{nopageno}
\pagestyle{empty}
\usepackage{color}


% Header, put stuff that impacts the whole document here
\NeedsTeXFormat{LaTeX2e}
\title{Introduction to Algorithms\\CSI 3344\\ Homework 0}
\author{Grant Gasser} 

\begin{document}
\maketitle 

Prepare a LaTex file with each problem beginning on a fresh page.
Use the Upload Site to submit your solution. These problems are a review of material from earlier courses that will serve us well in CSI 3344:

\begin{itemize}
	\item Using $big-O ~notation, and ~also ~\Omega ~and ~\Theta$
	\item Geometric series
	\item Simple proofs by induction 
	\item Logarithms and exponents 
\end{itemize}

\begin{enumerate}

\item Geometric series.
\begin{enumerate}
	\item Give a simple upper bound on $1+2+4+ \dots +2^n$. Conclude that this sum is $O(2^n)$.
	\item Do the same for $1+ \frac{1}{2 }+ \frac{1}{4} + \dots + \frac{1}{2^n} $ and conclude that the sum is O(1).
	\item By considering a more general series $S(n) = 1 + c + c^2 +  \dots + c^n$, establish the following very useful rule: in big-O terms, the sum of a geometric series is simply:
		\item Common ratio, r = c in this series.
\begin{enumerate}
\item the first term if the series is strictly decreasing, 
\item  the last term if the series is strictly increasing, 
\item  the number of terms if the series is unchanging.
\end{enumerate} 
\end{enumerate} 

\item Proving by induction. We would like to establish the following formula for the sum of the first n odd numbers: $1+3+5+ ... +(2n-1) = n^2. $ A nice way to do this is by induction. Let S(n) be the statement above. \textbf{We want to prove} $\sum_{i = 1}^{n}(2n-1) = n^2.$  

An inductive proof would have the following steps:
\begin{enumerate}
	
	 \item  Show that S(1) is true.
	 	\begin{enumerate}
	 		\item Let $n = k.$ That gives $(2(1)-1) = 1^2.$\\ $(2-1) = 1^2.$\\ $1 = 1.$ Thus, it is true for the base case.\\	
	 	\end{enumerate}
     \item Assume that if S(1),...,S(k) are true, 
     \item then show S(k+1) is true.
     	\begin{enumerate}
     		\item If it is true for S(1)...S(k), then $\sum_{i = 1}^{k}(2k-1) = k^2.$\\ 
     		\item Let n = k + 1. Then we have, $\sum_{i = 1}^{k+1}(2(k+1)-1) = (k+1)^2.$\\\\ $\sum_{i = 1}^{k+1}(2k+2-1) = k^2 + 2k + 1.$\\\\ $\sum_{i = 1}^{k+1}(2k+1) = k^2 + 2k + 1.$\\\\ $3 + 5 + 7 + .. + 2k-1 + 2(k+1)-1 = k^2 + 2k + 1.$\\\\ $\sum_{i = 1}^{k}(2k-1) + 2(k+1)-1 = k^2 + 2k + 1.$\\\\
     		$k^2 + 2(k+1)-1 = k^2 + 2k + 1.$ (Inductive hypothesis)\\\\ 
     		$k^2 + 2k + 1 = k^2 + 2k + 1.$ Thus, it is true for the base case and inductive hypothesis. QED.
     	\end{enumerate}
     

    
\end{enumerate}

\item Practice with $big-O ~and ~\Omega$. For some fixed positive integer c, consider the summation
$S(n)=1^c +2^c +3^c + \dots +n^c.$
\begin{enumerate}
	\item Show that S(n) is $O(n^{c+1})$. Hint: There are n terms in the series, and each is at most ...?
    \item Show that S(n) is $ \Omega(n^{c+1})$. Hint: Look just at the second half of the series.
\end{enumerate}
\item Logarithms (base two). Recall the definition of logarithm base two: saying $p = log_2 m$ is the same as saying
$m = 2^p$. In this class, we will typically write log to mean $log_2$.
\begin{enumerate}
\item How many bits are needed to write down a positive integer n? Give your answer in big-O notation,
as a function of n, where n is the length of the number.
\item How many times does the following piece of code print “hello”? Assume n is an integer, and that
division rounds down to the nearest integer. Give your answer in big-O form, as a function of n.

\begin{verbatim}
    while n > 1:
       print "hello"
       n := n/2
\end{verbatim}

\item In the following code, subroutine A(n) takes time $O(n^3)$. What is the overall running time of the loop, in big-O notation as a function of n? Assume that division by two takes linear time.
\begin{verbatim}
    while n > 1: 
        A(n)
        n := n/2
\end{verbatim}
\end{enumerate}

\item Logarithms (base b). Now we consider logarithms to base $b > 1$ : saying $p = log_b m$ is the same as saying
$m = b^p.$ The following transformation rule is helpful whenever switching between different bases: 

$log_a m = (log_a b)(log_b m).$

\begin{enumerate}
\item True or false: $log_2 n \in O(log_3 n)$? 
\item True or false: $2^{log_2 n} \in O(2^{log_3 n})$?
\item True or false: $(log_2 n)^2 \in O((log_3 n)^2)$?
\end{enumerate}


\item Minimal big-O notation. The following statements are all true:


$24n2+10n+20 \in  O(24n^2 + 10n + 20)$

$24n^2+10n+20 \in  O(24n^2)$

$24n^2+10n+20 \in  O(n^10)$

$24n^2+10n+20 \in  O(n^2)$


However, the last one is the simplest, cleanest, and tightest of them, and we will refer to it as the minimal big-O form. Write the following expressions in minimal big-O notation.

\begin{enumerate}
\item $100n^3 + 3^n.$
\item $200n~log(200n).$
\item $100n^22^n + 3^n.$
\item $100n~log(n) + 20n^3 + \sqrt{n}.$
\item $1^3 +2^3 + \dots +n^3.$
\end{enumerate}

In the examples above, the minimal form is unambiguous, but sometimes it is a matter of judgment. 

For instance, $5^{log_{2}n} = n^{log_2 5}$ can both reasonably be considered minimal (although we will tend to prefer the latter since it emphasizes that the function is polynomial).

\item A d-ary tree is a rooted tree in which each node has at most d children.
\begin{enumerate}
\item We can number the levels of the tree as 0,1,..., with level 0 consisting just of the root, level 1 consisting of its children, and so on. The largest level is the depth of the tree. Give a formula for the maximum possible number of nodes at level j of the tree, in terms of j and d.

\item Suppose the tree has depth k. Give a formula for the maximum possible number of nodes in the tree, in terms of k and d. You can leave your answer in big-O notation.

\item Suppose the tree has n nodes. What is the minimum the depth could possibly be, in terms of n and d? You can leave your answer in big-O format.
\end{enumerate}

\end{enumerate}
\end{document}






